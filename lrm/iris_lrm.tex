\documentclass{article}
\usepackage{graphicx} % Required for inserting images
\usepackage[a4paper, total={6in, 9in}]{geometry} %margins
\usepackage{multicol} %lists with multiple columns
\usepackage{xcolor}

\usepackage{hyperref} % creating hyperlinks in table of contents
\hypersetup{
    colorlinks = true,
    linkbordercolor = {white},
    allcolors = {blue}
}

%%%%%%%%%%%%%%%%%%%%%%%%%%%%%%%%%%%%%%%%%%%%%%%%%%%%%%%%%%%%%%%%%%%%%%%%%%%%%%%%%%%%%%%%%%%%
%
%
%                                   Document Begins
%
%
%%%%%%%%%%%%%%%%%%%%%%%%%%%%%%%%%%%%%%%%%%%%%%%%%%%%%%%%%%%%%%%%%%%%%%%%%%%%%%%%%%%%%%%%%%%%

\begin{document}
\title{The Iris Language Reference Manual}
\author{Valerie Zhang, Josh Kim, Ayda Aricanli, Tim Valk, and Trevor Sullivan}
\date{October 2023}
\maketitle

\pagebreak
\tableofcontents 

\pagebreak

\section{Introduction}
Iris is a general-purpose, object-oriented programming language that combines features from Java, C, and C++. The language is class-oriented in that \textit{almost} everything is a class. As such, it supports inheritance as well as other features like polymorphic, mutable lists, and encapsulation. \\

\noindent Iris introduces a spin on the normal conventions of encapsulation by introducing \texttt{permit} class members. These members are accessible by a class's sub-classes or other classes in its private collection of class names \texttt{permitted}. However, unlike \textit{friend} classes in C++, a class's permitted classes may only access the \texttt{permit} methods of the class rather than all class methods. 
\subsection{Notation}
Code is distinguished from prose by using this \texttt{font}.
\section{Lexical Conventions}
\subsection{Comments}
The characters \texttt{..} introduce a single-line comment. \newline
The characters \texttt{.\(\sim\)*} introduce a multi-line comment, which terminates with \texttt{*\(\sim\).}

\subsection{Identifier Conventions}
\noindent Any valid identifier in Iris is an uppercase or lowercase alphabetic character followed by any number of uppercase or lowercase alphabetic characters and/or digits. Uppercase and lowercase letters are distinct. The underscore \texttt{\_} is also an alphabetic character.

\subsection{Keywords}
\begin{multicols}{3}
\begin{itemize}
    \item[] \texttt{if}
    \item[] \texttt{else}
    \item[] \texttt{for}
    \item[] \texttt{while}
    \item[] \texttt{return}
    \item[] \texttt{class}
    \item[] \texttt{new}
    \item[] \texttt{of}
    \item[] \texttt{public}
    \item[] \texttt{permit}     % maybe different name?
    \item[] \texttt{permitted}
    \item[] \texttt{private}
    \item[] \texttt{void}
    \item[] \texttt{univ}
    \item[] \texttt{const}
    \item[] \texttt{int}
    \item[] \texttt{bool}
    \item[] \texttt{char}
    \item[] \texttt{string}
    \item[] \texttt{float}
    \item[] \texttt{List}       % brainstorm name
    \item[] \texttt{Olympus}    % keyword??
\end{itemize}
\end{multicols}
\subsection{Literals}
Iris allows for literals for each primitive data type:

\subsubsection{Integer literals}
An integer literal is any sequence of one or more digits. Integer literals can begin with exactly one "\texttt{-}", and if so, are assumed to be negative. All integer literals are assumed to be in base 10.

\subsubsection{Float literals}
A floating literal consists of 3 parts: a sequence of one or more digits, the decimal point, and another sequence of one or more digits. All 3 of these parts MUST be present for a float literal. Floating literals can optionally begin with exactly one "\texttt{-}", and if so, are assumed to be negative. All float literals are assumed to be in base 10.

\subsubsection{Boolean literals}
\texttt{true} and \texttt{false} are the only two boolean literals. These literals can be constructed from ASCII characters. 

\subsubsection{Character literals}
A character literal consists of one or two characters encapsulated in \texttt{'}. Inside the single quotes, there can be a single ASCII character, or one of the following special cases occurs: one of \texttt{'}, \texttt{n}, or \texttt{\textbackslash} is preceded by a \texttt{\textbackslash} to encode the single quote, newline, or backslash character, respectively.

\subsubsection{String literals}
A string literal consists of a series of character literals surrounded by double quotes. Within the string literal, a \texttt{"} character must be preceded by a single \texttt{\textbackslash}.
\section{What's In a Name?}
optional univ, datatype, identifier, 
\section{Types}

\subsection{Operators}

\section{Expressions}


\section{Classes}

\subsection{Encapsulation}

\subsection{Inheritance}



\section{Built-In Functions}

\subsection{Lists}

\subsection{The Olympus Class}

\section{Scope}

\end{document}

